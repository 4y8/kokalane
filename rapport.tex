\documentclass[11pt]{article}
\usepackage[utf8]{inputenc}
\usepackage[T1]{fontenc}
\usepackage[a4paper, margin=1.2in]{geometry}

\usepackage[backend=bibtex]{biblatex}
\addbibresource{rapport.bib}

\author{Aghilas Y. Boussaa}
\title{Premier rapport du compilateur mini-koka}
\begin{document}
\maketitle
\section{Le compilateur}
Le compilateur passe tous les tests d'analyse syntaxique, de typage et de
production de code.

\subsection{Analyse lexicale}
Pour l'analyse lexicale on implémente l'algorithme donné
dans l'énoncé, en utilisant une file permettant au lexer de renvoyer plusieurs
lexèmes en une fois.

\subsection{Analyse syntaxique}
Pour l'analyse syntaxique on modifie la grammaire donnée pour la rendre LR(1)
(en particulier, on n'utilise les directives d'associativités que pour les
opérateurs binaires). On a fait ce choix sur recommandation de François Pottier
donnée après un ``rapport de bug''\footnote{suite à l'utilisation de directives
menhir a renvoyé une erreur: \texttt{Conflict (unexplicable) [...] Please send
  your grammar to Menhir's developers}, il s'agit, d'après la réponse reçu par
mail, d'un problème connu que l'on peut éviter en donnant une grammaire
LR(1)}. Certaines constructions syntaxiques ont nécessités une attention
particulière:
\begin{itemize}
  \item les opérateurs binaires doivent traiter à part l'expression la plus à
    droite car elle peut être de la forme \texttt{return 1}, alors que
    \texttt{return 1 + 2} est interprété comme \texttt{return (1 + 2)},
  \item pour traiter le problème du sinon pendouillant, on distingue a une règle
    à part générant des expressions ne pouvant pas être suivie par des
    \texttt{else},
  \item pout traiter l'associativité à gauche des \texttt{fn} infixes on
    remarque qu'un atome a, à droite, soit un atome, soit un atome suivi du mot
    clé \texttt{fn} puis d'une expression se terminant par un bloc (une
    expression se terminant par un atome ou un bloc); on crée donc une règle
    pour les expressions se terminant par un bloc, ce qui nous permet de traiter
    ces deux cas.
\end{itemize}

\subsection{Typage}
Pour le typage on implémente l'algorithme J d'inférence de type sans la
généralisation ni l'instantiation, auquel on ajoute les effets. Pour les
fonctions de librairies standards qui ont du polymorphisme d'effet, on vérifie
que les types ont la bonne forme sans se soucier des effets.

\begin{figure}[h]\label{code2}
\begin{verbatim}
fun g(f : () -> <div> ())
  ()

fun f()
  g(f)
  println(42)
\end{verbatim}
\caption{un exemple nécessitant une variable d'effet}
\end{figure}

Le code de la figure~\ref{code2} montre que l'on ne peut se contenter de gérer
les effets de manière naïve avec seulement des ensembles, car il faut choisir un
type pour \texttt{f} dans \texttt{f}.

Les effets sont représentés comme un ensemble d'effet muni d'une référence
optionnelle vers un booléen optionnelle.

\begin{figure}[h]
\label{code1}
\begin{verbatim}
fun f()
  println(f())
  1
\end{verbatim}
\caption{les GADT à la rescousse}
\end{figure}

Le code de la figure~\ref{code1} montre que le typage ne peut pas être que
linéaire, car à la deuxième ligne, le typeur ne sais pas encore que \texttt{f()}
est affichable. Pour résoudre ce problème le typeur peut ajouter des contraintes
qui ne seront testées qu'après la première passe de typage (en même temps que
l'explicitation des fermetures), aussi bien pour l'affichage que pour les
opérations binaires polymorphes. On pourrait se contenter de garder des listes
globales de variable de types pour chaque ``classe de type'' et les vérifier à
la fin.

Cependant, les informations apportées par ces tests peuvent alléger les phases
suivantes si on les garde: on peut, par exemple, remplacer les appels à
\texttt{println} par des appels à des fonctions monomorphes. Pour faire cela, on
voudrait créer un constructeur \texttt{CheckConstraint of typed\_expr *
  typed\_expr -> typed\_expr}, où la fonction permet, par exemple, de
transformer l'expression \texttt{e} donnée en \texttt{println\_t(e)} si le type
\texttt{t} de \texttt{e} peut s'afficher.

En revanche, cela, ne marche pas pour les opérateurs binaires tels que
\texttt{++} pour lesquels on voudrait une fonction prenant deux expressions. On
pourrait mettre une liste en argument mais cela ajouterait du filtrage par motif
redondant. Les GADT nous permettent, enfin, d'avoir un constructeur
\texttt{\texttt{CheckConstraint : 'b * ('b -> typed\_expr) -> typed\_desc}},
sans que la variable de type n'apparaîssent dans le type des expressions.

\subsection{Production de code}
Après le typage, on explicite les fermetures et les positions des variables sur
la pile. Pour la génération de code on suit, pour l'instant, l'énoncé à la
différence que lorsqu'on appelle une fonction, un pointeur vers la fermeture est
placée dans le registre \texttt{\%r12}.

\section{Améliorations et extensions possibles}
On envisage plusieurs (plus ou moins ambitieuses) pistes pour continuer le
compilateur:
\begin{itemize}
  \item polymorphisme de type,
  \item ajout de filtrage par motif et de type de donnés algébriques,
  \item polymorphisme d'effet,
  \item libération de la mémoire des programmes, par exemple avec le comptage de
    références donné par l'algorithme perceus~\cite{reinking2021perceus} pour
    rester proche du vrai Koka,
  \item compilation optimisante comme vue en cours
\end{itemize}
\printbibliography
\end{document}
